%\documentclass[class=minimal,border=0pt]{standalone}
\documentclass[aspectratio=169]{beamer}
%\url{}
\usepackage{tikz}
\usepackage{ctex}
\definecolor{frenchblue}{HTML}{0071BB}
\definecolor{cursorred}{HTML}{FF0017}
\definecolor{cursorgrey}{HTML}{F1F1F1}
\definecolor{golden}{HTML}{FFBB3A}
\definecolor{grassgreen}{HTML}{62AD54}
\definecolor{ligthgray}{HTML}{D9D9D9}
\usepackage{comment}

%%%%%%%%%%%%%%%%%%%%%%%%%%%%%%%%%%%%%%%%%%%%%%%%


\makeatletter
\def\grd@save@target#1{%
  \def\grd@target{#1}}
\def\grd@save@start#1{%
  \def\grd@start{#1}}
\tikzset{
  grid with coordinates/.style={
    to path={%
      \pgfextra{%
        \edef\grd@@target{(\tikztotarget)}%
        \tikz@scan@one@point\grd@save@target\grd@@target\relax
        \edef\grd@@start{(\tikztostart)}%
        \tikz@scan@one@point\grd@save@start\grd@@start\relax
        \draw[minor help lines] (\tikztostart) grid (\tikztotarget);
        \draw[major help lines] (\tikztostart) grid (\tikztotarget);
        \grd@start
        \pgfmathsetmacro{\grd@xa}{\the\pgf@x/1cm}
        \pgfmathsetmacro{\grd@ya}{\the\pgf@y/1cm}
        \grd@target
        \pgfmathsetmacro{\grd@xb}{\the\pgf@x/1cm}
        \pgfmathsetmacro{\grd@yb}{\the\pgf@y/1cm}
        \pgfmathsetmacro{\grd@xc}{\grd@xa + \pgfkeysvalueof{/tikz/grid with coordinates/major step}}
        \pgfmathsetmacro{\grd@yc}{\grd@ya + \pgfkeysvalueof{/tikz/grid with coordinates/major step}}
        \foreach \x in {\grd@xa,\grd@xc,...,\grd@xb}
        \node[anchor=north] at (\x,\grd@ya) {\pgfmathprintnumber{\x}};
        \foreach \y in {\grd@ya,\grd@yc,...,\grd@yb}
        \node[anchor=east] at (\grd@xa,\y) {\pgfmathprintnumber{\y}};
      }
    }
  },
  minor help lines/.style={
    help lines,
    step=\pgfkeysvalueof{/tikz/grid with coordinates/minor step}
  },
  major help lines/.style={
    help lines,
    line width=\pgfkeysvalueof{/tikz/grid with coordinates/major line width},
    step=\pgfkeysvalueof{/tikz/grid with coordinates/major step}
  },
  grid with coordinates/.cd,
  minor step/.initial=.2,
  major step/.initial=1,
  major line width/.initial=2pt,
}

\makeatother
\begin{document}
\begin{tikzpicture}
%%%%%%%%%%%%%%%%%%%%%%%%%%%%%%%%%%%%%%%%%%%%%%%%
%%%%%coordination system below
%\begin{comment}
\draw[help lines,step=.2] (-2,-4) grid (10,4);
\draw[help lines,line width=.6pt,step=1] (-2,-4) grid (10,4);
\foreach \x in {-2,-1,0,1,2,3,4,5,6,7}
 \node[anchor=north] at (\x,-2) {\x};
\foreach \y in {-4,-3,-2,-1,0,1,2,3,4}
 \node[anchor=east] at (-2,\y) {\y};
%\end{comment}
%%%%%%%%%%%%%%%%%%%%%%%%%%%%%%%%%%%%%%%%%%%%%%%%
%%%%%your code below

\node (contents) at (10,5) [anchor=west,fill=frenchblue,draw=frenchblue,text=white] {\Large{大纲}\hspace*{20ex}};

\node (chapIntro) at (0,4) [anchor=west,fill=frenchblue,draw=frenchblue, text=white] {1.引言};
\node (chapK2Social) at (0,2.8) [anchor=west,fill=golden,draw=golden, text=black] {2.基于知识迁移的社交媒体事件检测通用框架K2Social};
\node (chapTransDetector) at (0,1.6) [anchor=west,fill=grassgreen,draw=grassgreen, text=white] {3.基于知识库结构的主题抽取与迁移方法};
%\node (chapUMIETM) at (0,0.4) [anchor=west,fill=frenchblue,draw=frenchblue, text=white] {4.用户兴趣建模与知识迁移方法};
%\node (chapTransDetector+) at (0,-0.8) [anchor=west,fill=frenchblue,draw=frenchblue, text=white] {5.基于知识迁移的领域事件检测通用方法};
\node (chapConclusion) at (0,-2) [anchor=west,fill=ligthgray,draw=ligthgray, text=frenchblue] {基于知识迁移的社交媒体事件检测通用框架K2Social};

\draw[color=frenchblue,-] (-1,4)--(chapIntro);
\draw[color=frenchblue,-] (-1,2.8)--(chapK2Social);
\draw[color=frenchblue,-] (-1,1.6)--(chapTransDetector);
%\draw[color=frenchblue,-] (-1,0.4)--(chapUMIETM);
%\draw[color=frenchblue,-] (-1,-0.8)--(chapTransDetector+);
\draw[color=frenchblue,-] (-1,-2)--(chapConclusion);
\draw[color=frenchblue,-] (-1,-2)--(-1,4);

%cursors
\draw[fill=cursorgrey, draw=cursorgrey, opacity=0.85, shift={(0.1,-0.1)}] (-1.1,4.2) -- (-1.1,3.8) -- (-0.15,3.8) -- (0,4) -- (-0.15,4.2) -- (-1.1,4.2);
\draw[fill=cursorred, draw=cursorred] (-1.1,4.2) -- (-1.1,3.8) -- (-0.15,3.8) -- (0,4) -- (-0.15,4.2) -- (-1.1,4.2);
%%%%%%%%%%%%%%%%%%%%%%%%%%%%%%%%%%%%%%%%%%%%%%%%

\end{tikzpicture}

\end{document}
